% Canevas pour la these, fruit du labeur d'une longue lignee de
% thesard-e-s inconnu-e-s. 

% Pour compiler : latex, latex, biber, latex, latex
\documentclass[12pt,a4paper]{book}

% ----------- Langue 
\usepackage[french]{babel}
%\usepackage[T1]{utf8}
\usepackage[T1]{fontenc}
% remplacer latin1 par utf8 en fonction de l'encodage de votre fichier
\usepackage[utf8]{inputenc}

% ----------- Organisation
\usepackage[french]{minitoc} % pour construire des mini tables des matieres

%------------ Mise en forme
\usepackage{fancyhdr}
\usepackage[fancyhdr]{styleThese}  % style de these, avec notamment les entetes
\usepackage{setspace} % pour pouvoir regler l'interligne
\usepackage{url}% pour les url
\usepackage[pdftex]{color}
\usepackage[usenames,dvipsnames,svgnames,table]{xcolor} %% To color in violet the text
\definecolor{bordeau}{rgb}{0.3515625,0,0.234375} % la couleur de Paris-Saclay
\usepackage[pdftex]{graphicx}
\usepackage[absolute]{textpos} % to place elements in the page
\usepackage{calc} % To calculate textwidth
\usepackage{pgf} % Images: pdf, png, jpg, mais pas eps
\usepackage{textcomp} % pour que 'ー' soit plus joli en francais
% pour avoir des liens dans la table des matieres et sur les citations
\usepackage[pdftex]{hyperref}
% pour faire corriger les liens hypertextes, qui pointent par d伺aut le bas d'unefigure ou d'un tableau
\usepackage{hypcap}
\usepackage{geometry}% That nicely create a one-page template
\usepackage{tikz} %% Make the square.
\usepackage{afterpage} % pour inserer une commande après le pagebreak suivant
\usepackage{csquotes} % pour utiliser les "guillemets" corrects en fonction de la langue

%------------ Objets specifiques
\usepackage{multicol} % pour pouvoir faire des cases de plusieurs colonnes dans les tableaux 
\usepackage{multirow} % pour pouvoir faire des cases de plusieurs lignes dans les tableaux 
\usepackage{supertabular} % pour les tableaux a cheval sur plusieurs pages
\usepackage{listings} % pour afficher du code

%------------ Pour la biblio
\usepackage[backend=biber,citestyle=authoryear,sorting=none,defernumbers]{biblatex}
\addbibresource{bibliothese.bib} % les ref biblio pour la thèse
\addbibresource{mespublications.bib} % les ref de mes publications
\defbibheading{bibempty}{} % pour gérer le niveau de titre de la biblio avec \chapter, \section, ...
%\DeclareRefcontext{myarticles}{sorting=ynt,labelprefix="J"}

% On utilise biblatex (ci-dessus), plutot que bibtex, pour avoir une biblio séparée des papiers publiés
% pour faire des citations type Dupont (1978), avec \citet et \citep, voir http://www.ctan.org/pkg/natbib
%\usepackage{natbib}
% \usepackage{fnamed}% idem que la mise en forme named.bst de natbib mais en francais.

%------------ Mathematiques
\usepackage{amsmath,amssymb}


%---------------------------------------------------
%----------- COMMANDES ----------------------------
%---------------------------------------------------

\definecolor{lnkcol}{rgb}{0,0,0.93}
\definecolor{extcol}{rgb}{0.33,0,0.55}
\hypersetup{
%    bookmarks=true,         % show bookmarks bar?
%     unicode=true,           % non-Latin characters in Acrobat's bookmarks
    pdftoolbar=true,        % show Acrobat's toolbar?
    pdfmenubar=true,        % show Acrobat's menu?
    pdffitwindow=true,      % page fit to window when opened
    pdftitle={Titre},    % title
    pdfauthor={Harry Cover},     % author
    pdfsubject={sujet},   % subject of the document
    pdfnewwindow=true,      % links in new window
    pdfkeywords={keywords}, % list of keywords
    colorlinks=true,        % false: boxed links; true: colored links
    linkcolor=lnkcol,       % color of internal links
    citecolor=lnkcol,       % color of links to bibliography
    filecolor=magenta,      % color of file links
    urlcolor=extcol         % color of external links
}

\setlength{\oddsidemargin}{0pt} \setlength{\evensidemargin}{0pt}
\setlength{\marginparwidth}{0pt} \setlength{\marginparsep}{0pt}
\parskip=6pt plus 2pt minus 3pt

% Reglage de la profondeur de la table des matieres
% 2 = on affiche les chapitres, les sections et les sous-sections
\setcounter{tocdepth}{2}

% Page blanche qui ne compte pas dans la numérotation
\newcommand\blankpage{%
    \null
    \thispagestyle{empty}%
    \addtocounter{page}{-1}%
    \newpage}

% Interligne
%\renewcommand{\baselinestretch}{1.3} \normalsize
\setstretch{1.3}

%---------------------------------------------------
%----------- MACROS -------------------------
%---------------------------------------------------
\newcommand{\tmop}[1]{\ensuremath{\operatorname{#1}}}
\newcommand{\Reel}{\mathbb{R}}
% Titre de la thèse
\newcommand{\PhDTitle}{Thesis title. It can extend over several lines (even 4 or 5)} 

% Prénom Nom
\newcommand{\PhDname}{Harry Cover} 

%---------------------------------------------------------------------
%--- DEBUT DU DOCUMENT -------------------------------
%---------------------------------------------------------------------

\begin{document}
 
% ---------- Pour tout le document ----------
% pour que des mini tables des matieres soient creees
\dominitoc
% pour que latex ne deborde pas dans la marge de droite
\sloppy
% -----------------------------------


% ------------------------------------------------------------------
% ---------- Page de garde ----------------------------------
% ------------------------------------------------------------------
\input{these_garde_upsay_v2}
%\cleardoublepage
\color{black}
\clearpage

% -------------------------------------------------------------------
% -- Debut du document -------------------------------------
% -------------------------------------------------------------------
% les pages sont numerotees en romain
\pagestyle{empty}
\frontmatter

%-------------------------------------------------------------------
% --------- En-tetes du debut--------------------------------
%-------------------------------------------------------------------
% pour les pages standard
\pagestyle{fancy}
\fancyhf{}
% on met juste le numero de page en bas de page
% (C=Center, E=Even (pair), O=Odd (impair))
\fancyfoot[CE,CO]{\thepage}
\renewcommand{\headrulewidth}{0pt}
% et pour les 1eres pages de chapitres
\fancypagestyle{plain}{
  \fancyhf{}
  \fancyfoot[CE,CO]{\thepage}
  \renewcommand{\headrulewidth}{0pt}
}

%-------------------------------------------------------------------
% ---------- Resume -------------------------------------------
%-------------------------------------------------------------------
%\pagestyle{plain}
\section*{Résumé}
\input{these_resume}

\section*{Abstract}
But I must explain to you how all this mistaken idea of denouncing pleasure and praising pain was born and I will give you a complete account of the system, and expound the actual teachings of the great explorer of the truth, the master-builder of human happiness. 

On the other hand, we denounce with righteous indignation and dislike men who are so beguiled and demoralized by the charms of pleasure of the moment, so blinded by desire, that they cannot foresee the pain and trouble that are bound to ensue; and equal blame belongs to those who fail in their duty through weakness of will, which is the same as saying through shrinking from toil and pain.

%-------------------------------------------------------------------
% ---------- Table des matieres et autres ----------------
%-------------------------------------------------------------------
\tableofcontents 
\listoffigures 
\listoftables

%-------------------------------------------------------------------
% ---------- Remerciements ---------------------------------
%-------------------------------------------------------------------
\chapter*{Remerciements}

Je tiens à remercier mes rapporteurs X et Y pour avoir accepté de siéger dans le jury de cette thèse.
Blah, blah , blah.

Je souhaite remercier mon directeur de thèse, Z, pour blah, blah.
J'adresse également à l'ensemble des membres du laboratoire XZ mes remerciements pour blah.

Évidemment, je tiens également tout particulièrement à remercier Blorg et Puck pour blah, blah.


%-------------------------------------------------------------------
% -------- Coeur du manuscrit ------------------------------
%-------------------------------------------------------------------
% la numerotation des pages recommence et est en chiffres arabes
\mainmatter
%-------------------------------------------------------------------
%                   Introduction
%-------------------------------------------------------------------
\chapter*{Introduction}
\addstarredchapter{Introduction}

\section*{Lorem ipsum}

Lorem ipsum dolor sit amet, consectetur adipiscing elit. Etiam commodo consectetur pellentesque. Nullam mollis tristique quam sit amet adipiscing. Integer id mauris sapien, quis egestas nibh. Sed lobortis ante nec urna gravida elementum. Sed dolor nibh, congue et scelerisque id, lobortis quis turpis. Morbi gravida semper euismod. Integer gravida mattis felis vitae porta. Vestibulum ultricies gravida metus, in venenatis nisi rutrum in. Phasellus accumsan sodales velit, id porttitor ipsum pulvinar et. Vivamus tempor est mauris, et eleifend dui. 

\section*{Contexte du travail}

Pellentesque ornare, leo dictum pretium aliquam, nisi elit sagittis neque, fermentum tempus nisl erat at turpis. Aliquam erat volutpat. Etiam pellentesque suscipit lorem non scelerisque. Donec id metus neque. Praesent tempor dui id mauris semper vehicula. Maecenas eget semper ante. Mauris quis ligula quis tellus pharetra aliquet nec at ante. Nulla enim justo, venenatis quis consectetur sit amet, posuere quis augue. Nulla facilisi. Aenean eget mi eu turpis euismod gravida. Aenean arcu lectus, scelerisque non iaculis id, venenatis ac neque. Aliquam mauris nulla, dictum sit amet faucibus sit amet, varius ut purus. Phasellus interdum placerat risus id molestie. Duis dictum lectus in justo imperdiet commodo. Donec sapien nulla, gravida non mattis sed, malesuada quis dolor. 

\section*{Probl\'ematique}

 Nunc fermentum, odio eget fermentum varius, metus tellus auctor tellus, et sagittis orci risus nec eros. Morbi dapibus, elit at fermentum bibendum, felis felis scelerisque elit, sed consequat libero urna eget lorem. Maecenas placerat quam sed turpis facilisis volutpat. Nullam aliquet justo vel nulla varius ac lobortis tortor pharetra. Aenean lobortis, odio volutpat dictum dictum, massa nisl eleifend enim, auctor placerat leo tortor id sem. Nullam sollicitudin magna at nunc suscipit sodales a sit amet erat. Aenean vitae facilisis lorem. Cum sociis natoque penatibus et magnis dis parturient montes, nascetur ridiculus mus. Praesent consectetur, turpis sed pellentesque sollicitudin, elit odio semper felis, quis malesuada enim ipsum nec velit. Maecenas consequat, nisl sed euismod accumsan, arcu dolor fermentum magna, vitae suscipit felis ante feugiat ligula. Integer commodo auctor mauris et varius. Maecenas id urna massa. In nec quam elit. Curabitur a libero leo, a pulvinar nisi.

\section*{Principales contributions}

 Nulla pellentesque viverra magna vel egestas. Ut venenatis, tellus a blandit congue, magna nibh porttitor lacus, a tincidunt leo urna sit amet turpis. Integer congue, enim eu scelerisque porttitor, neque nulla sagittis diam, ac venenatis arcu magna vel massa. Pellentesque sagittis sodales odio, et placerat mauris venenatis sodales. Pellentesque non pretium sapien. Donec laoreet, urna nec ultrices congue, odio leo aliquet nunc, sit amet ullamcorper metus sapien sed quam. Lorem ipsum dolor sit amet, consectetur adipiscing elit. Nam quis massa tellus. Vivamus eleifend faucibus justo in luctus. Proin lacinia, ante dapibus aliquet viverra, quam neque mattis enim, bibendum vehicula felis lorem sed erat.

\section*{Plan de la th\`ese}

 Etiam tristique orci a tellus facilisis adipiscing lacinia leo placerat. Fusce eu tortor eget est placerat pulvinar eget non odio. Nullam in ligula ipsum. Aliquam condimentum egestas elementum. Suspendisse justo nisl, sagittis at fermentum vel, feugiat nec tellus. Vivamus vitae justo lectus, non eleifend ligula. Suspendisse imperdiet, dui eget semper fermentum, justo augue rhoncus lorem, id convallis dolor odio a ante. Ut eget leo ut ante cursus ultricies.


\thispagestyle{plain}

%-------------------------------------------------------------------
% -------- En-tetes a partir du 1er chapitre --------------
%-------------------------------------------------------------------

\pagestyle{fancy}

\fancyhf{}
%\markboth : contient le nom du chapitre courant tel qu'il apparait dans la table des matieres.
%(et donc avec des minuscules, alors que par defaut, fancy le met en majuscules)
\renewcommand{\chaptermark}[1]{\markboth{#1}{}}
%\markright : contient le nom de la section courante telle qu'elle apparait dans la table des matieres.
\renewcommand{\sectionmark}[1]{\markright{#1}}
%bas de pages
\fancyfoot[CE,CO]{\thepage}
%pages paires
%\thechapter = numero du chapitre courant et \leftmark = nom du chapitre courant
\fancyhead[EL]{Chapitre \thechapter~- \leftmark}
%pages impaires
%\thesection = numero de la section courante et \rightmark = nom de la section courante
\fancyhead[OR]{\thesection. \rightmark}

\renewcommand{\headrulewidth}{0.5pt}

%-------------------------------------------------------------------
%--------- Chapitres--------------------------------------------
%-------------------------------------------------------------------

\chapter[État de l'art]{État de l'art}
\minitoc

Nunc ultricies commodo eros, ut vestibulum risus convallis et \cite{ALE97}. Nunc tempus metus non tellus aliquet quis aliquet enim interdum. Aliquam ante ipsum, mattis at accumsan non, condimentum non magna. Duis et mi quam, et consequat neque. Proin convallis \parencite{BRE03,CHA03} tincidunt erat vel vestibulum. Maecenas adipiscing, mauris quis facilisis malesuada, diam risus posuere felis, quis scelerisque erat lorem eu arcu \parencite[voir][page 12]{BRE03}. Nunc tempor elit nec ligula tincidunt dapibus. Nulla facilisi. 

\section{Phasellus quis ipsum}

Sed dolor massa, cursus sit amet semper et, lobortis at nisl. Donec tellus massa, gravida \cite{SHE04} sit amet volutpat vel, vestibulum sed eros. Phasellus quis ipsum mauris, in elementum lorem. Fusce metus mi, tincidunt id congue eget, euismod eu libero. Aenean et lacus sit amet nibh malesuada placerat. Fusce dictum diam non massa cursus elementum \cite{ARB98}. Nulla facilisi. Duis a sem odio, et tincidunt magna. Phasellus vitae purus at erat tristique tempor. Maecenas ut accumsan augue. Integer facilisis tellus nec nunc pellentesque id mattis lacus rhoncus. Praesent a eros a urna interdum adipiscing sed quis turpis. 

\subsection{Aenean ut nulla libero}

Etiam nisl lectus, suscipit id sodales ac, venenatis non orci. Aliquam id ligula et arcu hendrerit fringilla. Aenean non erat magna, id fringilla arcu. Vestibulum pellentesque lacinia eros ac aliquet. Nulla sapien lacus, malesuada semper tincidunt vitae, lacinia vel felis. Fusce molestie pretium gravida. 

\subsection{Maecenas sem nisl}

Maecenas sem nisl, sodales quis suscipit et, imperdiet non nulla. Proin dolor ipsum, placerat tincidunt dapibus ac, pellentesque non massa. Aliquam erat volutpat. Aenean dolor velit, ultrices ut tristique ac, ultrices a mi. Aenean non magna non tellus tristique rutrum \cite{FOL98}. Vivamus consectetur viverra lorem, eu mollis tortor aliquet in. Quisque ut eleifend nunc. Ut accumsan purus nec sapien egestas congue. Vestibulum blandit orci leo. Sed viverra tempus nibh, non adipiscing augue semper id. Pellentesque id ligula vel ipsum sagittis bibendum non id neque. Donec pharetra, turpis quis laoreet hendrerit, nisi massa ultrices turpis, quis lacinia sapien nunc vitae erat. Etiam eu risus arcu. Donec vitae elit vel leo pretium rhoncus. 

\section{In dolor justo}

In dolor justo, ullamcorper non euismod posuere, dapibus quis mi. Ut feugiat ipsum pulvinar eros fermentum at egestas est ultricies. Morbi sit amet \cite{AMA75} mi quis ligula pellentesque lacinia. Nulla id velit mollis nulla semper hendrerit eget sit amet magna. Donec dui neque, blandit vitae venenatis quis, malesuada id tellus. Nulla facilisi. Nunc id erat tortor, vitae luctus enim. Phasellus feugiat commodo vehicula. Vestibulum scelerisque porttitor dui, ut placerat leo fringilla in. Etiam molestie hendrerit lacinia. Etiam iaculis vulputate tempor. Suspendisse molestie urna eget justo auctor tempus ut quis diam. Maecenas bibendum vehicula viverra. Donec nec libero purus, non venenatis purus. 

\subsection{Sed sollicitudin}

Sed sollicitudin, mauris quis interdum tincidunt, dui sem porttitor eros, ut mattis tortor sapien at eros. Aenean luctus mauris vitae diam fringilla ac blandit est iaculis. Curabitur ut tempor enim. Morbi accumsan sodales nulla hendrerit ullamcorper \cite{LAI01}. Ut faucibus imperdiet felis vitae egestas. Aliquam non turpis eget dui convallis egestas ac ut lacus. Pellentesque leo justo, blandit sed tempor blandit, aliquet non nunc. Phasellus sit amet venenatis elit. 

\subsection{Duis in luctus massa}

Nullam condimentum mollis leo, quis tincidunt lorem malesuada vitae. Maecenas a cursus neque. Suspendisse id urna orci, in egestas nibh. Quisque justo velit, dapibus id imperdiet non, condimentum et neque. Lorem ipsum dolor sit amet, consectetur adipiscing elit. Ut mollis erat nec urna iaculis at interdum est rhoncus. Pellentesque sed urna metus. Duis eros mi, feugiat a tincidunt at, rhoncus eget tellus. Maecenas tristique gravida vehicula. Quisque molestie lacinia orci, id mattis arcu eleifend non. Ut vel augue sit amet metus egestas fringilla. 



\chapter{Mon problème}\minitoc

\section{Class aptent taciti}

 Class aptent taciti sociosqu ad litora torquent per conubia nostra, per inceptos himenaeos. Etiam ullamcorper, arcu in feugiat vulputate, metus odio sodales mauris, sed suscipit nunc tortor ut quam. Aliquam mattis rhoncus nulla, in rutrum urna blandit eu. Vivamus dui justo, sagittis at pharetra vitae, bibendum at nulla. Ut ullamcorper felis quis ipsum scelerisque id ultrices nunc rhoncus. Suspendisse eu nisi non orci pharetra dignissim. Phasellus placerat dignissim eros quis hendrerit. Etiam eget nisl nec lectus feugiat egestas at sed neque. Vestibulum posuere erat at urna sagittis sed feugiat lorem elementum.


\subsection{In at orci diam}

Donec nisl felis, eleifend ut commodo nec, congue eu nisi. Ut laoreet dignissim velit, quis sollicitudin nisl pellentesque eget. Nam enim tellus, aliquet at pharetra sed, placerat consectetur lorem. Pellentesque vulputate sollicitudin dictum. Nulla eget ante facilisis eros suscipit vulputate a vitae nulla. Vestibulum ante ipsum primis in faucibus orci luctus et ultrices posuere cubilia Curae; Maecenas rhoncus iaculis varius. Sed ullamcorper ligula a nisl venenatis non dignissim tellus posuere. Nullam id arcu at nisl ullamcorper hendrerit viverra ullamcorper nulla. Vivamus suscipit faucibus turpis, dapibus interdum purus egestas quis. Aliquam dictum imperdiet metus a blandit. Aliquam erat volutpat. Sed ac neque at tortor egestas congue. Quisque enim lorem, pellentesque quis ultricies eu, condimentum eget purus. Nam hendrerit justo a tortor vestibulum ac semper nibh lacinia. 

\subsection{Sed vehicula}

Sed vehicula, diam in pretium posuere, risus enim sodales urna, egestas faucibus enim justo fringilla nisi. In hac habitasse platea dictumst. Nulla eros dui, sollicitudin eget ornare vitae, feugiat vel orci. Curabitur in dolor est. Ut molestie velit at metus placerat imperdiet. Nulla justo ligula, rhoncus eget pellentesque eget, sagittis et neque. Sed tincidunt volutpat rhoncus. Phasellus ut magna eu magna consectetur malesuada. Phasellus molestie accumsan nisi, eu mattis tortor egestas sit amet. Nunc in malesuada magna. Vivamus elementum porta ligula, id bibendum risus bibendum id. Integer ornare tristique dictum. 

\section{Suscipit nunc tortor}

 Class aptent taciti sociosqu ad litora torquent per conubia nostra, per inceptos himenaeos. Etiam ullamcorper, arcu in feugiat vulputate, metus odio sodales mauris, sed suscipit nunc tortor ut quam. Aliquam mattis rhoncus nulla, in rutrum urna blandit eu. Vivamus dui justo, sagittis at pharetra vitae, bibendum at nulla. Ut ullamcorper felis quis ipsum scelerisque id ultrices nunc rhoncus. Suspendisse eu nisi non orci pharetra dignissim. Phasellus placerat dignissim eros quis hendrerit. Etiam eget nisl nec lectus feugiat egestas at sed neque. Vestibulum posuere erat at urna sagittis sed feugiat lorem elementum.


\subsection{Aliquam dictum imperdiet}

Donec nisl felis, eleifend ut commodo nec, congue eu nisi. Ut laoreet dignissim velit, quis sollicitudin nisl pellentesque eget. Nam enim tellus, aliquet at pharetra sed, placerat consectetur lorem. Pellentesque vulputate sollicitudin dictum. Nulla eget ante facilisis eros suscipit vulputate a vitae nulla. Vestibulum ante ipsum primis in faucibus orci luctus et ultrices posuere cubilia Curae; Maecenas rhoncus iaculis varius. Sed ullamcorper ligula a nisl venenatis non dignissim tellus posuere. Nullam id arcu at nisl ullamcorper hendrerit viverra ullamcorper nulla. Vivamus suscipit faucibus turpis, dapibus interdum purus egestas quis. Aliquam dictum imperdiet metus a blandit. Aliquam erat volutpat. Sed ac neque at tortor egestas congue. Quisque enim lorem, pellentesque quis ultricies eu, condimentum eget purus. Nam hendrerit justo a tortor vestibulum ac semper nibh lacinia. 

\subsection{Vivamus elementum}

Sed vehicula, diam in pretium posuere, risus enim sodales urna, egestas faucibus enim justo fringilla nisi. In hac habitasse platea dictumst. Nulla eros dui, sollicitudin eget ornare vitae, feugiat vel orci. Curabitur in dolor est. Ut molestie velit at metus placerat imperdiet. Nulla justo ligula, rhoncus eget pellentesque eget, sagittis et neque. Sed tincidunt volutpat rhoncus. Phasellus ut magna eu magna consectetur malesuada. Phasellus molestie accumsan nisi, eu mattis tortor egestas sit amet. Nunc in malesuada magna. Vivamus elementum porta ligula, id bibendum risus bibendum id. Integer ornare tristique dictum. 



\include{these_chap-mesexpes}

\chapter{Conclusions et perspectives}
\section{Conclusions}

 Sed ligula nisi, mollis a varius vitae, auctor et ligula. Vestibulum malesuada ultricies nisl, posuere hendrerit felis vulputate a. Aliquam mattis odio sit amet nunc ultricies adipiscing. Maecenas et sem felis, in vestibulum est. Cras ac nulla tellus. Morbi sodales, nulla sit amet consequat sollicitudin, sem justo vulputate eros, vitae blandit enim justo vel turpis. Phasellus mattis nunc quis nulla accumsan ac auctor lacus consequat. Duis quam felis, varius tincidunt facilisis ut, congue auctor ante. Sed non vehicula dui. Class aptent taciti sociosqu ad litora torquent per conubia nostra, per inceptos himenaeos. Morbi aliquet malesuada laoreet. Fusce erat enim, ultricies vitae posuere molestie, tempus vitae dui. Cum sociis natoque penatibus et magnis dis parturient montes, nascetur ridiculus mus. Nullam viverra feugiat tincidunt. Sed nec tincidunt est.

Quisque ac nisi et turpis ullamcorper viverra at nec massa. Aenean porta eros enim, a commodo tellus. Vestibulum vitae nulla est, vitae condimentum leo. Vestibulum eget tortor et nunc lobortis pulvinar. Sed vitae justo quam, egestas rhoncus mauris. Mauris sed orci non turpis tempor faucibus. Maecenas ultricies imperdiet justo, ac facilisis est sagittis ut. Nam at vestibulum quam. Aliquam erat volutpat. Proin tincidunt arcu ac nisi tincidunt ac hendrerit enim facilisis. Donec nibh nisl, hendrerit quis tempor vel, vehicula eget risus. Vestibulum ante ipsum primis in faucibus orci luctus et ultrices posuere cubilia Curae; In vehicula lectus a felis malesuada luctus. Pellentesque eu purus enim, vitae fringilla dui.

\section{Perspectives}

 Cras odio mi, vehicula sit amet viverra et, tincidunt id mauris. In eu ornare libero. Aliquam sit amet est nunc, vel rhoncus urna. Sed turpis dui, dignissim eget porttitor a, pulvinar eu mi. In hac habitasse platea dictumst. Vivamus eu tortor sit amet leo blandit porta. Maecenas rutrum mollis justo, ac lacinia urna imperdiet at. Nulla condimentum rutrum eros sit amet porta. Duis a convallis nisl. Etiam elementum sapien quis lorem euismod nec malesuada neque varius.

Aenean scelerisque scelerisque sem vitae ultrices. In hac habitasse platea dictumst. Maecenas porttitor nulla non purus aliquam volutpat quis eget tortor. Nulla scelerisque varius lobortis. Etiam non mollis urna. Morbi ut purus nisl. Suspendisse eu molestie dui. Praesent porttitor, tellus vel interdum ultrices, velit nibh aliquam nunc, vitae aliquam mauris ipsum id dolor. Nam ac consequat lectus. Maecenas ac pulvinar tellus. Maecenas justo lorem, faucibus in pulvinar vitae, aliquam at eros. Nunc non arcu in lorem egestas tincidunt non eu augue. Praesent mauris erat, consequat id tincidunt quis, posuere a risus. Integer ac nulla id quam suscipit scelerisque non eu tortor. Donec posuere orci sed quam cursus mollis. Etiam non fringilla nisi. 

%-------------------------------------------------------------------
%--------- Bibliographie --------------------------------------
%-------------------------------------------------------------------

% --------- En-tetes de la fin (= debut) --------------------
% pour les pages standard
\pagestyle{fancy}
\fancyhf{}
% on met juste le numero de page en bas de page
% (C=Center, E=Even (pair), O=Odd (impair))
\fancyfoot[CE,CO]{\thepage}
\renewcommand{\headrulewidth}{0pt}
% et pour les 1eres pages de chapitres
\fancypagestyle{plain}{
  \fancyhf{}
  \fancyfoot[CE,CO]{\thepage}
  \renewcommand{\headrulewidth}{0pt}
}

\chapter*{Publications}
\label{chap:mespubli}
\addstarredchapter{Publications}

\begin{refsection} 

\newrefcontext[labelprefix=J]
\nocite{*}
\printbibliography[keyword=journal,resetnumbers=true,heading=subbibliography,title={Journaux internationaux avec comité de relecture}]

\newrefcontext[labelprefix=CI]
\nocite{*}
\printbibliography[keyword=confinternationale, resetnumbers=true,heading=subbibliography,title={Conférences internationales avec comité de relecture}]

\newrefcontext[labelprefix=CN]
\nocite{*}
\printbibliography[keyword=confnationale,resetnumbers=true,heading=subbibliography,title={Conférence nationale avec comité de relecture}]
\end{refsection} 



  \if@twoside
    \cleardoublepage
  \else
    \clearpage
  \fi

\addstarredchapter{Bibliographie}

% pour que la biblio ne soit pas enorme
\begin{singlespace}
    \printbibliography 
\end{singlespace}

\appendix

\if@twoside
    \cleardoublepage
\else
    \clearpage
\fi

% remettre a 0 tous les compteurs de fig, tab et eq.
\renewcommand{\theequation}{A-\arabic{equation}}
\setcounter{equation}{0}  % reset counter 
\renewcommand{\thefigure}{A-\arabic{figure}}
\setcounter{figure}{0}  % reset counter 
\renewcommand{\thetable}{A-\arabic{table}}
\setcounter{table}{0}  % reset counter 

% comme les annexes peuvent etres longues, on remet les bordures de pages
\pagestyle{fancy}

\fancyhf{}
%\markboth : contient le nom du chapitre courant tel qu'il apparait dans la table des matieres.
%(et donc avec des minuscules, alors que par defaut, fancy le met en majuscules)
\renewcommand{\chaptermark}[1]{\markboth{#1}{}}
%\markright : contient le nom de la section courante telle qu'elle apparait dans la table des matieres.
\renewcommand{\sectionmark}[1]{\markright{#1}}
%bas de pages
\fancyfoot[CE,CO]{\thepage}
%pages paires
%\thechapter = numero du chapitre courant et \leftmark = nom du chapitre courant
\fancyhead[EL]{Chapitre \thechapter~- \leftmark}
%pages impaires
%\thesection = numero de la section courante et \rightmark = nom de la section courante
\fancyhead[OR]{\thesection. \rightmark}
\renewcommand{\headrulewidth}{0.5pt}

\include{these_annexe-demonstrations}

\chapter{Fusce sagittis}
\section{Morbi vitae sapien}

Donec mollis, dolor non pharetra volutpat, ipsum mi interdum tellus, ut pharetra ipsum eros a tortor. Morbi a dapibus dui. Duis sollicitudin scelerisque turpis non tempor. Maecenas at arcu justo, et facilisis augue. Nulla semper mollis tempor. Nulla interdum, nulla sit amet eleifend mattis, orci eros malesuada turpis, a sagittis lectus dolor id odio. Integer ultricies vehicula iaculis. Mauris in turpis at augue ornare rutrum sed in ipsum. Nulla facilisi. Nunc vehicula, tortor at ornare dictum, ligula urna bibendum nisl, a pellentesque est urna non lacus. Mauris congue vulputate velit, vel faucibus erat malesuada sit amet. 

\section{Ut facilisis}

Suspendisse pulvinar bibendum velit, et rutrum erat malesuada in. Donec a erat augue, condimentum aliquet nisi. Phasellus iaculis consequat rhoncus. Ut venenatis, purus ut euismod rutrum, eros mi bibendum ipsum, sit amet luctus dolor ipsum sit amet nibh. Nulla molestie quam vitae erat rhoncus in eleifend erat luctus. Ut placerat massa sit amet neque venenatis non mattis leo tincidunt. Sed neque lorem, laoreet a auctor vitae, suscipit eu massa. Maecenas placerat pharetra purus sed ornare. Proin turpis neque, suscipit at placerat quis, sodales dictum velit. 


\input{these_lastpage.tex}
\end{document}


